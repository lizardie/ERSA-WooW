\documentclass[ignorenonframetext,]{beamer}
\usetheme{Darmstadt}
\usecolortheme{beaver}
\usefonttheme{structurebold}
\setbeamertemplate{caption}[numbered]
\setbeamertemplate{caption label separator}{:}
\setbeamercolor{caption name}{fg=normal text.fg}
\usepackage{amssymb,amsmath}
\usepackage{ifxetex,ifluatex}
\usepackage{fixltx2e} % provides \textsubscript
\usepackage{lmodern}
\ifxetex
  \usepackage{fontspec,xltxtra,xunicode}
  \defaultfontfeatures{Mapping=tex-text,Scale=MatchLowercase}
  \newcommand{\euro}{€}
\else
  \ifluatex
    \usepackage{fontspec}
    \defaultfontfeatures{Mapping=tex-text,Scale=MatchLowercase}
    \newcommand{\euro}{€}
  \else
    \usepackage[T1]{fontenc}
    \usepackage[utf8]{inputenc}
      \fi
\fi
% use upquote if available, for straight quotes in verbatim environments
\IfFileExists{upquote.sty}{\usepackage{upquote}}{}
% use microtype if available
\IfFileExists{microtype.sty}{\usepackage{microtype}}{}

% Comment these out if you don't want a slide with just the
% part/section/subsection/subsubsection title:
\AtBeginPart{
  \let\insertpartnumber\relax
  \let\partname\relax
  \frame{\partpage}
}
\AtBeginSection{
  \let\insertsectionnumber\relax
  \let\sectionname\relax
  \frame{\sectionpage}
}
\AtBeginSubsection{
  \let\insertsubsectionnumber\relax
  \let\subsectionname\relax
  \frame{\subsectionpage}
}

\setlength{\parindent}{0pt}
\setlength{\parskip}{6pt plus 2pt minus 1pt}
\setlength{\emergencystretch}{3em}  % prevent overfull lines
\setcounter{secnumdepth}{0}

\title{Advanced Research Tools for Economics and Business Administration (Part
II)}
\author{Thomas de Graaff}
\date{January 22, 2015}

\begin{document}
\frame{\titlepage}

\section{Introduction}\label{introduction}

\begin{frame}{Previous tutorial}

Still somewhat more theoretical (why do you want to change tools)

\begin{itemize}
\item
  Importance of writing things down (reproducability)
\item
  Text files are the bomb:

  \begin{itemize}
  \itemsep1pt\parskip0pt\parsep0pt
  \item
    scriptable
  \item
    input and output in/for other applications
  \end{itemize}
\item
  pros and cons of \LaTeX
\end{itemize}

\end{frame}

\begin{frame}{This tutorial}

More practical, play around with \LaTeX. In specific:

\begin{itemize}
\itemsep1pt\parskip0pt\parsep0pt
\item
  packages (make things look better)
\item
  figures (usually import them, but sometime make them yourself)
\item
  tables (import them!)
\item
  slides (just copy \& paste from \texttt{.tex} document)
\end{itemize}

\end{frame}

\section{Making appearances}\label{making-appearances}

\begin{frame}{The use of packages}

There are lots of them, see \href{http://www.ctan.org}{CTAN}

\end{frame}

\begin{frame}{The use of classes}

\end{frame}

\begin{frame}{Bibliopgraphy}

\end{frame}

\section{Better graphs}\label{better-graphs}

\begin{frame}{Import them}

\end{frame}

\begin{frame}{Making them yourself}

\end{frame}

\section{Better tables}\label{better-tables}

\begin{frame}{Some guidelines}

\end{frame}

\begin{frame}{Importing them}

\end{frame}

\section{Making slides}\label{making-slides}

\begin{frame}{Pros and cons}

\end{frame}

\begin{frame}{Using beamer package}

\end{frame}

\end{document}
