\documentclass[ignorenonframetext,]{beamer}
\usetheme{Darmstadt}
\usecolortheme{beaver}
\usefonttheme{structurebold}
\setbeamertemplate{caption}[numbered]
\setbeamertemplate{caption label separator}{:}
\setbeamercolor{caption name}{fg=normal text.fg}
\usepackage{amssymb,amsmath}
\usepackage{ifxetex,ifluatex}
\usepackage{fixltx2e} % provides \textsubscript
\usepackage{lmodern}
\ifxetex
  \usepackage{fontspec,xltxtra,xunicode}
  \defaultfontfeatures{Mapping=tex-text,Scale=MatchLowercase}
  \newcommand{\euro}{€}
\else
  \ifluatex
    \usepackage{fontspec}
    \defaultfontfeatures{Mapping=tex-text,Scale=MatchLowercase}
    \newcommand{\euro}{€}
  \else
    \usepackage[T1]{fontenc}
    \usepackage[utf8]{inputenc}
      \fi
\fi
% use upquote if available, for straight quotes in verbatim environments
\IfFileExists{upquote.sty}{\usepackage{upquote}}{}
% use microtype if available
\IfFileExists{microtype.sty}{\usepackage{microtype}}{}
\usepackage{color}
\usepackage{fancyvrb}
\newcommand{\VerbBar}{|}
\newcommand{\VERB}{\Verb[commandchars=\\\{\}]}
\DefineVerbatimEnvironment{Highlighting}{Verbatim}{commandchars=\\\{\}}
% Add ',fontsize=\small' for more characters per line
\usepackage{framed}
\definecolor{shadecolor}{RGB}{248,248,248}
\newenvironment{Shaded}{\begin{snugshade}}{\end{snugshade}}
\newcommand{\KeywordTok}[1]{\textcolor[rgb]{0.13,0.29,0.53}{\textbf{{#1}}}}
\newcommand{\DataTypeTok}[1]{\textcolor[rgb]{0.13,0.29,0.53}{{#1}}}
\newcommand{\DecValTok}[1]{\textcolor[rgb]{0.00,0.00,0.81}{{#1}}}
\newcommand{\BaseNTok}[1]{\textcolor[rgb]{0.00,0.00,0.81}{{#1}}}
\newcommand{\FloatTok}[1]{\textcolor[rgb]{0.00,0.00,0.81}{{#1}}}
\newcommand{\CharTok}[1]{\textcolor[rgb]{0.31,0.60,0.02}{{#1}}}
\newcommand{\StringTok}[1]{\textcolor[rgb]{0.31,0.60,0.02}{{#1}}}
\newcommand{\CommentTok}[1]{\textcolor[rgb]{0.56,0.35,0.01}{\textit{{#1}}}}
\newcommand{\OtherTok}[1]{\textcolor[rgb]{0.56,0.35,0.01}{{#1}}}
\newcommand{\AlertTok}[1]{\textcolor[rgb]{0.94,0.16,0.16}{{#1}}}
\newcommand{\FunctionTok}[1]{\textcolor[rgb]{0.00,0.00,0.00}{{#1}}}
\newcommand{\RegionMarkerTok}[1]{{#1}}
\newcommand{\ErrorTok}[1]{\textbf{{#1}}}
\newcommand{\NormalTok}[1]{{#1}}

% Comment these out if you don't want a slide with just the
% part/section/subsection/subsubsection title:
\AtBeginPart{
  \let\insertpartnumber\relax
  \let\partname\relax
  \frame{\partpage}
}
\AtBeginSection{
  \let\insertsectionnumber\relax
  \let\sectionname\relax
  \frame{\sectionpage}
}
\AtBeginSubsection{
  \let\insertsubsectionnumber\relax
  \let\subsectionname\relax
  \frame{\subsectionpage}
}

\setlength{\parindent}{0pt}
\setlength{\parskip}{6pt plus 2pt minus 1pt}
\setlength{\emergencystretch}{3em}  % prevent overfull lines
\setcounter{secnumdepth}{0}

\title{Advanced Research Tools for Economics and Business Administration (Part
II)}
\author{Thomas de Graaff}
\date{January 22, 2015}

\begin{document}
\frame{\titlepage}

\section{Introduction}\label{introduction}

\begin{frame}{Previous tutorial}

Still somewhat more theoretical (why do you want to change tools)

\begin{itemize}
\item
  Importance of writing things down (reproducability)
\item
  Text files are the bomb:

  \begin{itemize}
  \itemsep1pt\parskip0pt\parsep0pt
  \item
    scriptable
  \item
    input and output in/for other applications
  \end{itemize}
\item
  pros and cons of \LaTeX
\end{itemize}

\end{frame}

\begin{frame}[fragile]{A quick recap}

\begin{itemize}
\item
  Specific \LaTeX~commands starts with an \textbackslash{}

  \begin{itemize}
  \itemsep1pt\parskip0pt\parsep0pt
  \item
    \texttt{\textbackslash{}LaTeX}
  \end{itemize}
\item
  Inline equations are within \$ \$

  \begin{itemize}
  \itemsep1pt\parskip0pt\parsep0pt
  \item
    \texttt{\$\textbackslash{}frac\{a\}\{b\}\$ is the fraction between \$a\$ and \$b\$}
  \end{itemize}
\item
  There are a number of symbols that you cannot immediately use:

  \begin{itemize}
  \itemsep1pt\parskip0pt\parsep0pt
  \item
    \textbackslash{}, \$, \&, \%, \{ and \} are the most important
    (solution: start with an \texttt{\textbackslash{}})
  \end{itemize}
\item
  Environments start and end

\begin{Shaded}
\begin{Highlighting}[]
\NormalTok{\textbackslash{}begin\{equation\} }
\NormalTok{a^2 + b^2 = c^2 }
\NormalTok{\textbackslash{}end\{equation\}}
\end{Highlighting}
\end{Shaded}
\end{itemize}

\end{frame}

\begin{frame}[fragile]{General structure}

\begin{Shaded}
\begin{Highlighting}[]
\NormalTok{\textbackslash{}documentclass[twocolumn, a4paper]\{article\}}

\CommentTok{% Preamble: how should it look like}
\NormalTok{\textbackslash{}usepackage\{multicol, lipsum\}}
\NormalTok{\textbackslash{}usepackage[english, german]\{babel\}}

\NormalTok{\textbackslash{}begin\{document\}}
    \CommentTok{% Body: the real contents}
    \NormalTok{\textbackslash{}lipsum}
\NormalTok{\textbackslash{}end\{document\}}
\end{Highlighting}
\end{Shaded}

\end{frame}

\begin{frame}{This tutorial}

More practical, play around with \LaTeX. In specific:

\begin{itemize}
\itemsep1pt\parskip0pt\parsep0pt
\item
  packages (make things look better)
\item
  figures (usually import them, but sometime make them yourself)
\item
  tables (import them!)
\item
  slides (just copy \& paste from \texttt{.tex} document)
\end{itemize}

\end{frame}

\section{Making appearances}\label{making-appearances}

\begin{frame}[fragile]{The use of packages}

\begin{itemize}
\item
  Typically, packages are used to change appearance
\item
  There are lots of them, see \href{http://www.ctan.org}{CTAN}
\item
  Often used packages

  \begin{itemize}
  \itemsep1pt\parskip0pt\parsep0pt
  \item
    amsmath, graphicx, subfig, marvosym, microtype, booktabs, lipsum,
    pdflscape, fullpage
  \end{itemize}
\item
  format:

\begin{Shaded}
\begin{Highlighting}[]
\NormalTok{\textbackslash{}usepackage\{amsmath, graphicx\}}
\end{Highlighting}
\end{Shaded}
\end{itemize}

\end{frame}

\begin{frame}{The use of classes}

\end{frame}

\begin{frame}{Bibliopgraphy}

\end{frame}

\section{Better graphs}\label{better-graphs}

\begin{frame}{Import them}

\end{frame}

\begin{frame}{Making them yourself}

\end{frame}

\section{Better tables}\label{better-tables}

\begin{frame}{Some guidelines}

\end{frame}

\begin{frame}{Importing them}

\end{frame}

\section{Making slides}\label{making-slides}

\begin{frame}{Pros and cons}

\end{frame}

\begin{frame}{Using beamer package}

\end{frame}

\end{document}
